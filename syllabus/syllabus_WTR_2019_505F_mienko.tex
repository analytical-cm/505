\documentclass[11pt,]{article}
\usepackage[margin=1in]{geometry}
\newcommand*{\authorfont}{\fontfamily{phv}\selectfont}
\usepackage[]{mathpazo}
\usepackage{abstract}
\renewcommand{\abstractname}{}    % clear the title
\renewcommand{\absnamepos}{empty} % originally center
\newcommand{\blankline}{\quad\pagebreak[2]}

\providecommand{\tightlist}{%
  \setlength{\itemsep}{0pt}\setlength{\parskip}{0pt}} 
\usepackage{longtable,booktabs}

\usepackage{parskip}
\usepackage{titlesec}
\titlespacing\section{0pt}{12pt plus 4pt minus 2pt}{6pt plus 2pt minus 2pt}
\titlespacing\subsection{0pt}{12pt plus 4pt minus 2pt}{6pt plus 2pt minus 2pt}

\titleformat*{\subsubsection}{\normalsize\itshape}

\usepackage{titling}
\setlength{\droptitle}{-.25cm}

%\setlength{\parindent}{0pt}
%\setlength{\parskip}{6pt plus 2pt minus 1pt}
%\setlength{\emergencystretch}{3em}  % prevent overfull lines 

\usepackage[T1]{fontenc}
\usepackage[utf8]{inputenc}

\usepackage{fancyhdr}
\pagestyle{fancy}
\usepackage{lastpage}
\renewcommand{\headrulewidth}{0.3pt}
\renewcommand{\footrulewidth}{0.0pt} 
\lhead{}
\chead{}
\rhead{\footnotesize SOC W 505 F: Foundations of Social Welfare Research (Analytical Case
Management) -- Winter 2019}
\lfoot{}
\cfoot{\small \thepage/\pageref*{LastPage}}
\rfoot{}

\fancypagestyle{firststyle}
{
\renewcommand{\headrulewidth}{0pt}%
   \fancyhf{}
   \fancyfoot[C]{\small \thepage/\pageref*{LastPage}}
}

%\def\labelitemi{--}
%\usepackage{enumitem}
%\setitemize[0]{leftmargin=25pt}
%\setenumerate[0]{leftmargin=25pt}




\makeatletter
\@ifpackageloaded{hyperref}{}{%
\ifxetex
  \usepackage[setpagesize=false, % page size defined by xetex
              unicode=false, % unicode breaks when used with xetex
              xetex]{hyperref}
\else
  \usepackage[unicode=true]{hyperref}
\fi
}
\@ifpackageloaded{color}{
    \PassOptionsToPackage{usenames,dvipsnames}{color}
}{%
    \usepackage[usenames,dvipsnames]{color}
}
\makeatother
\hypersetup{breaklinks=true,
            bookmarks=true,
            pdfauthor={ ()},
             pdfkeywords = {},  
            pdftitle={SOC W 505 F: Foundations of Social Welfare Research (Analytical Case
Management)},
            colorlinks=true,
            citecolor=blue,
            urlcolor=blue,
            linkcolor=magenta,
            pdfborder={0 0 0}}
\urlstyle{same}  % don't use monospace font for urls


\setcounter{secnumdepth}{0}





\usepackage{setspace}

\title{SOC W 505 F: Foundations of Social Welfare Research (Analytical Case
Management)}
\author{Joe Mienko}
\date{Winter 2019}


\begin{document}  

		\maketitle
		
	
		\thispagestyle{firststyle}

%	\thispagestyle{empty}


	\noindent \begin{tabular*}{\textwidth}{ @{\extracolsep{\fill}} lr @{\extracolsep{\fill}}}


E-mail: \texttt{\href{mailto:mienko+505@uw.edu}{\nolinkurl{mienko+505@uw.edu}}} & Web: \href{http://TBD}{\tt TBD}\\
Office Hours: By Appointment  &  Class Hours: F 830-1120\\
Office: TBD  & Class Room: SWS 036\\
	&  \\
	\hline
	\end{tabular*}
	
\vspace{2mm}
	


\section{Course Description}\label{course-description}

This course is an overview of research process/methods in social work,
with focus on consuming and performing practice-related research and
evaluating one's own practice. Emphasis on critical understanding of
empirical literature, development of useful and appropriate questions
about social work practice, and strategies and techniques for doing
research and applying findings to practice.

This is a pilot course which has been modified from the traditional
research methods sequence in two important ways: 1. The course is
designed to be focused on the application of analytical tools to direct
social work practice, and 2. The course is strongly oriented toward
child welfare social work.

Given the pilot nature of the course, the specific schedule and
assignments outlined in this syllabus are subject to change. Any changes
or additions to this syllabus will be made with at least 1-week notice
to students.

\section{Course Objectives}\label{course-objectives}

\begin{enumerate}
\def\labelenumi{\arabic{enumi}.}
\tightlist
\item
  To introduce students to a variety of research methodologies,
  epistemologies, and ethical considerations that accompany research
  endeavors. This will include exposure to qualitative and quantitative
  methodologies and human participant considerations.
\item
  To learn to critically assess social science research from ethical,
  multicultural, and social justice perspectives.
\item
  To gain a balanced and informed perspective about varying types of
  research methodologies and epistemological traditions.
\item
  To identify the potential limitations of research methods and examine
  critically conclusions drawn from data in the practice setting and
  from the research literature.
\item
  To gain experience in the direct application of research skills to
  define, measure, and conceptualize problems of relevance to social
  work practice.
\end{enumerate}

\section{Required Readings}\label{required-readings}

To be provided in hard copy or digital form as required.

\section{Targeted Competencies and Related Practice
Behaviors}\label{targeted-competencies-and-related-practice-behaviors}

This course targets the following Council on Social Work Accreditation
(CSWE) competencies and related practice behaviors:

\subsection{Competencies}\label{competencies}

\subsubsection{Competency \#2: Apply social work ethical principles to
guide professional
practice.}\label{competency-2-apply-social-work-ethical-principles-to-guide-professional-practice.}

\begin{itemize}
\tightlist
\item
  Make ethical decisions, in practice and in research, by critically
  applying the ethical standards of the NASW Code of Ethics and other
  relevant codes of ethics.
\end{itemize}

\subsubsection{Competency \#3: Apply critical thinking to inform and
communicate professional
judgments.}\label{competency-3-apply-critical-thinking-to-inform-and-communicate-professional-judgments.}

\begin{itemize}
\tightlist
\item
  Use critical thinking to distinguish, evaluate, and integrate multiple
  sources of knowledge, including research-based knowledge, practice
  wisdom, and client/constituent experience.
\item
  Critically analyze models of assessment, especially in relation to
  their cultural relevance and applicability and their promotion of
  social justice.
\item
  Critically analyze models of prevention, especially in relation to
  their cultural relevance and applicability and their promotion of
  social justice.
\item
  Critically analyze models of intervention, especially in relation to
  their cultural relevance and applicability and their promotion of
  social justice.
\item
  Critically analyze models of evaluation, especially in relation to
  their cultural relevance and applicability and their promotion of
  social justice.
\end{itemize}

\subsubsection{Competency \#5: Advance human rights and social and
economic
justice.}\label{competency-5-advance-human-rights-and-social-and-economic-justice.}

\begin{itemize}
\tightlist
\item
  Advocate for and engage in practices that address disparities and
  inequalities and advance human rights and social and economic justice.
\end{itemize}

\subsubsection{Competency \#6: Engage in research-informed practice and
practice-informed
research.}\label{competency-6-engage-in-research-informed-practice-and-practice-informed-research.}

\begin{itemize}
\tightlist
\item
  Use client and constituent knowledge to inform research and
  evaluation.
\item
  Use my own practice experience to inform research and evaluation.
\item
  Use qualitative research evidence to inform practice.
\item
  Use quantitative research evidence to inform practice.
\item
  Apply research literature on social disparities when selecting and
  evaluating services and policies.
\end{itemize}

\subsubsection{Competency \#10: Engage, assess, intervene, and evaluate
with individuals, families, groups, organizations, and
communities.}\label{competency-10-engage-assess-intervene-and-evaluate-with-individuals-families-groups-organizations-and-communities.}

\begin{itemize}
\tightlist
\item
  Collect, organize, and interpret client/constituent/system data
  (e.g.~strengths, stressors, and limitations) to assess
  client/constituent needs.
\item
  Critically analyze, monitor, and evaluate interventions.
\end{itemize}

\subsection{Learning Outcomes - On successful completion of the course
you will be able
to:}\label{learning-outcomes---on-successful-completion-of-the-course-you-will-be-able-to}

\begin{itemize}
\tightlist
\item
  Have a working knowledge of a variety of research methodologies,
  epistemologies, and ethical considerations that accompany research
  endeavors.
\item
  Be able to evaluate research with a critical eye on studies that can
  perpetuate or counter oppression against vulnerable populations.
\item
  Be able to refine research design and articulate data collection and
  analytic procedures that reflect practice assessment, implementation,
  and outcome issues.
\item
  Be able to critically evaluate empirical evidence to promote
  evidence-informed social work practice and policy.
\end{itemize}

\section{Instructional Methods}\label{instructional-methods}

Education research shows that your learning is greatest when you are
actively involved in making sense of new concepts (``constructing
knowledge'') and when you do this in social settings. We will use this
model throughout the course, so you can expect to

\begin{itemize}
\tightlist
\item
  be engaged in plenty of classroom activities to build on the readings
  you have done for each class
\item
  work in small groups during class and for those groups to change on a
  regular basis
\item
  ask your instructor for clarifications, rather than expecting
  lectures.
\end{itemize}

Many of our courses will involve non-lecture material which seek to
practically orient you to the concepts in the readings and to help you
understand how the theoretical concepts covered in the readings relate
to practice. If you find that you haven't managed to complete a reading
before class, you will likely find that particular class frustrating,
since we will build on and apply the readings each time (including
trouble-shooting the issues you found most perplexing). I hope you find
this an engaging and enjoyable approach to learning.

\section{Course Policy}\label{course-policy}

I will detail the policy for this course below. Basically, don't cheat
and try to learn stuff. Don't be that person.

\subsection{Classroom Norms}\label{classroom-norms}

All conduct in the classroom will be professional with an emphasis on
maintaining an environment that is mutually respectful and that supports
the educational process. These norms include, but are not limited to:

\begin{itemize}
\tightlist
\item
  Start and end on time
\item
  Come to class prepared
\item
  Participate actively in discussion
\item
  Show one another courtesy, including when we disagree.
\end{itemize}

\subsection{Confidentiality}\label{confidentiality}

Discussion of case material may occur in class. All classroom
discussions about case material or an individual participant's personal
experiences are considered confidential and may not be discussed or
otherwise shared outside the classroom.

\subsection{Grading Policy (See Table 1 for the Specific Grading
Scale)}\label{grading-policy-see-table-1-for-the-specific-grading-scale}

\begin{itemize}
\item
  \textbf{10\%} of your grade will be based on your participation in a
  post test assessing your knowledge at end of this quarter. This test
  will be graded on a credit/no-credit basis. The test will be posted
  online within the first two weeks of the quarter. The test will be
  posted online during the final exam week.
\item
  \textbf{20\%} of your grade will be determined by an ``evidence-based
  investigative assessment'' that you will complete based on data,
  assessment tools, and other information provided by me.
\item
  \textbf{60\%} of your grade will be determined by a series of
  practical exercises which will help apply the knowledge you've
  gathered in lecture during a given week.
\item
  \textbf{10\%} of your grade will be determined by your attendance and
  participation in class. Generally, ask questions and answer them.
\end{itemize}

\subsection{Textual Description of Letter
Grades}\label{textual-description-of-letter-grades}

\begin{itemize}
\tightlist
\item
  A/A- Mastery of subject content, demonstration of critical analysis,
  creativity and/or complexity in completion of the assignment. The
  difference between an A and an A- is based on the degree to which
  these competencies are demonstrated.
\item
  B+ Mastery of subject content beyond expected competency, but lacking
  in additional critical analysis, creativity, or complexity in the
  completion of the assignment.
\item
  B Mastery of subject content at level of expected competency; meets
  course expectations.
\item
  B- Less than adequate competency, but demonstrates student learning
  and potential for mastery of subject content.
\item
  C-/C+ Demonstrates a minimal understanding of subject content.
  Significant areas need improvement to meet course expectations.
\item
  D/E Failure to demonstrate minimal understanding of subject content.
\end{itemize}

\subsection{Bibliography and Citation
Requirements}\label{bibliography-and-citation-requirements}

When required, all citations must follow the APA Publication Manual (6th
edition), since it is the standard referencing system for Social Work.
It may be different from other systems you have used, so follow the
Manual's citation guidelines carefully. This is an opportunity to
demonstrate your attention to detail.

\subsection{Assignment Deadlines and
Extensions}\label{assignment-deadlines-and-extensions}

In this class, you are expected to conduct yourselves as professional,
courteous, and well-organized individuals -- this is what any
organization will expect of you when you complete your degrees. Acting
in this way helps give UW graduates a reputation as excellent and
reliable colleagues, and in turn it means that your degree is worth more
in a competitive marketplace. One of the most important ways you will
demonstrate these behaviors in class is by ensuring that your work is
ALWAYS ON TIME.

Assignments must be submitted by the set deadlines and will typically be
returned within 5 business days. It is essential that you plan ahead for
all eventualities to ensure that none of your work is late. Check the
session-by-session schedule at the end of this syllabus to see when
assignments. Briefings about each assignment will occur during a
preceding class. This enables you to plan now. Block out time in your
calendar now so that you know exactly when you will be working on
assignments for this course. Make sure you give yourself extra time just
in case you run into difficulty with an assignment, have a computer
problem, or feel unwell.

Late assignments are not accepted without prior discussion and approval.
When accepted, late assignments (even if same day) receive an automatic
10\% grade reduction, increasing 10\% per day late. Retakes or ``make up
work'' for failing grades are not offered.

\subsection{Student Responsibilities for
Learning}\label{student-responsibilities-for-learning}

You can expect to devote an average of two hours outside of class to the
subject matter (readings and preparation, as well as substantive
assignments and participation exercises) for every hour in class. As
this is a three-credit class, you can reasonably expect an average of 6
hours of homework each week. I have tried to ensure that the workload is
evenly distributed throughout the course, but if you find you have less
than the normal amount of work one week, I suggest you read ahead for
future classes. Please refer to other course policies on attendance,
participation, missed classes, and assignment deadlines earlier in this
syllabus.

\begin{table}[]
\centering
\caption{Grading Scale}
\label{my-label}
\begin{tabular}{lll}
Numeric Grade-Point Equivilant & Letter Grade & Points \\
4.0                           & A            & 100    \\
4.0                           & A            & 99     \\
3.9                           & A            & 98     \\
3.9                           & A            & 97     \\
3.8                           & A-           & 96     \\
3.8                           & A-           & 95     \\
3.7                           & A-           & 94     \\
3.7                           & A-           & 93     \\
3.6                           & A-           & 92     \\
3.6                           & A-           & 91     \\
3.5                           & A-           & 90     \\
3.5                           & A-           & 89     \\
3.4                           & B+           & 88     \\
3.3                           & B+           & 87     \\
3.3                           & B+           & 86     \\
3.2                           & B+           & 85     \\
3.1                           & B+           & 84     \\
3.0                           & B            & 83     \\
2.9                           & B            & 82     \\
2.8                           & B-           & 81     \\
2.7                           & B-           & 80     \\
2.6                           & B-           & 79     \\
2.5                           & B-           & 78     \\
2.4                           & C+           & 77     \\
2.3                           & C+           & 76     \\
2.2                           & C+           & 75     \\
2.1                           & C+           & 74     \\
2.0                           & C            & 73     \\
1.9                           & C            & 72     \\
1.8                           & C            & 71     \\
1.7                           & C            & 70     \\
1.6-0.0                       & F            & 69     \\
\\
\textit{2.7 is lowest passing grade for a required course}
\end{tabular}
\end{table}

\subsection{Attendance Policy}\label{attendance-policy}

\begin{quote}
\emph{Showing up is 80 percent of life} -- Woody Allen
\end{quote}

Students should be weary of skipping class. Attendance at all class
sessions is expected. Missing a class session or coming late or leaving
early without having talked to the instructor will detract from the
participation grade. If you are going to miss all or part of a class due
to illness or another unavoidable commitment I expect you to call and
leave a message on my phone or e-mail me. Students who may need
accommodations for disability-related absences should discuss these
arrangements ahead of time with the instructor. More information about
disability-related absences at:
\url{http://depts.washington.edu/uwdrs/accommodations/disability-related-absence/}

\subsection{E-mail Policy}\label{e-mail-policy}

I check my email regularly, which is the best way to reach me. You
should note the alias (``+505'') that I am asking you to use for this
quarter at the top of the syllabus. Adding this to my email will help me
filter your emails amongst the 100-200 messages I receive every day. You
can email me at any time, but you may not receive a response outside
regular business hours. Generally, emails received before 3 p.m. will
receive a response before 5 p.m., and emails received after 3 p.m. will
receive a response on the following business day. Business days are
Monday--Friday, except for holidays.

\subsection{Academic Dishonesty
Policy}\label{academic-dishonesty-policy}

Don't cheat. Don't be that person. Yes, you. You know exactly what I'm
talking about too.

\section{Academic Resources}\label{academic-resources}

My goal is to create a learning environment in which you can be
successful. I will work hard to create and improve the learning
environment throughout the quarter based on my own observations of the
course and your feedback on what would help you learn more. In return, I
ask and encourage you to make the most of this learning opportunity.
Please take advantage of the academic support services available to you
at the university. Even if you have had excellent study skills in the
past, it is easy to slip into suboptimal habits and these services can
help you excel in your studies.

\subsection{\texorpdfstring{SSW Librarian - Angela Lee,
\href{mailto:leea@uw.edu}{\nolinkurl{leea@uw.edu}},
685-2180}{SSW Librarian - Angela Lee, leea@uw.edu, 685-2180}}\label{ssw-librarian---angela-lee-leeauw.edu-685-2180}

\subsection{\texorpdfstring{SSW Writing Tutors -
\url{http://socialwork.uw.edu/students/services/writing-tutors}}{SSW Writing Tutors - http://socialwork.uw.edu/students/services/writing-tutors}}\label{ssw-writing-tutors---httpsocialwork.uw.edustudentsserviceswriting-tutors}

\subsection{Support for Students with
Disabilities}\label{support-for-students-with-disabilities}

At the SSW we are committed to ensuring access to classes, course
material, and learning opportunities for students with disabilities.
Your experience in this class is important to us, and it is the policy
and practice of the University of Washington to create inclusive and
accessible learning environments consistent with federal and state law.
If you experience barriers based on a disability or temporary health
condition, please seek a meeting with DRS to discuss and address them.
If you have already established accommodations with DRS, please
communicate your approved accommodations to your instructor at your
earliest convenience so we can discuss your needs in this course.
Disability Resources for Students (DRS) offers resources and coordinates
reasonable accommodations for students with disabilities and/or
temporary health conditions. Reasonable accommodations are established
through an interactive process between you, your instructor(s) and DRS.
If you have not yet established services through DRS, but have a
temporary health condition or permanent disability that requires
accommodations (this can include but not limited to; mental health,
attention-related, learning, vision, hearing, physical or health
impacts), you are welcome to contact DRS at 206-543-8924 or
\href{mailto:uwdrs@uw.edu}{\nolinkurl{uwdrs@uw.edu}} or
disability.uw.edu

\subsection{Self-care / Counseling}\label{self-care-counseling}

It is assumed that participants in the class may have had their own
lives or the lives of individuals they know touched in some way by
trauma. Class discussions, presentations, readings or other classroom
events may trigger strong emotions. Students are encouraged to engage in
self-care, which may include leaving the classroom, without explanation,
at any time, as needed. Students are welcome to discuss personal
reactions in class, but are in no way required or expected to do so.
Students are also encouraged to speak with the instructor at non-class
times about any such reactions. Free support resources are also
available at the Student Counseling Center.

\subsection{Counseling Resources}\label{counseling-resources}

The UW Counseling Center offers free and confidential short-term,
problem focused counseling to UW Students who may feel overwhelmed by
the responsibilities of school, work, family and relationships.
Counselors are available to help students cope with stresses and
personal issues that may interfere with their ability to perform in
school. To schedule an appointment, please call 206-543-1240 or stop by
401 Schmitz Hall. More information at:
\url{http://www.washington.edu/counseling/} If you're looking for
additional low-cost resources, we've posted a list at the bottom of this
page: \url{http://socialwork.uw.edu/node/4339}

\newpage

\section{Class Schedule}\label{class-schedule}

Students must read the following before Friday's class session.
Important: class readings are subject to change, contingent on
mitigating circumstances and the progress we make as a class. Students
are encouraged to attend lectures and check the course website for
updates.

\subsection{Week 01, 01/07 - 01/11: Syllabus
Day}\label{week-01-0107---0111-syllabus-day}

\subsection{Week 02, 01/14 - 01/18: No Class This
Week!}\label{week-02-0114---0118-no-class-this-week}

\emph{social work researchers have a conference this week}

Read the following documents and come prepared to discuss next week:

\begin{itemize}
\tightlist
\item
  \href{https://www.socialworkers.org/About/Ethics/Code-of-Ethics/Code-of-Ethics-English}{NASW
  Code of Ethics}
\item
  Gary King, Robert O Keohane, and Sidney Verba. (1994). \emph{Designing
  Social Inquiry: Scientific Inference in Qualitative Research.}
  Princeton: Princeton University Press.
\end{itemize}

Complete the following housekeeping activities prior to our next
meeting:

\begin{itemize}
\tightlist
\item
  \href{https://csde.washington.edu/netid/account/new.php/}{Register for
  a CSDE Account}
\item
  \href{https://www.uptodate.com/}{Explore UpToDate}
\end{itemize}

\subsection{Week 03, 01/21 - 01/25: Social Service Science and
Measurement}\label{week-03-0121---0125-social-service-science-and-measurement}

\begin{itemize}
\tightlist
\item
  No readings this week
\end{itemize}

\subsection{Week 04, 01/28 - 02/01: Investigative Assessment - Failure
to Thrive and Caloric
Intake}\label{week-04-0128---0201-investigative-assessment---failure-to-thrive-and-caloric-intake}

\begin{itemize}
\item
  Olsen, Else Marie, Janne Petersen, Anne Mette Skovgaard, Birgitte
  Weile, Torben Jørgensen, and Charlotte M. Wright. \emph{Failure to
  thrive: the prevalence and concurrence of anthropometric criteria in a
  general infant population.} Archives of disease in childhood 92, no. 2
  (2007): 109-114.
\item
  Review ``Estimated energy requirements'' for children from UpToDate.
\item
  Review ``Growth charts'' for children from UpToDate.
\end{itemize}

\subsection{Week 05, 02/04 - 02/08: Investigative Assessment - Drug and
Alcohol
Screening}\label{week-05-0204---0208-investigative-assessment---drug-and-alcohol-screening}

\begin{itemize}
\item
  Keary, Christopher J., et al. \emph{Toxicologic testing for opiates:
  understanding false-positive and falsenegative test results.} The
  primary care companion for CNS disorders 14.4 (2012).
\item
  Brahm, Nancy C., et al. CLINICAL CONSULTATION. \emph{Commonly
  prescribed medications and potential false-positive urine drug
  screens.} American Journal of Health-System Pharmacy 67.16 (2010).
\item
  Humeniuk, Rachel, et al. \emph{Validation of the alcohol, smoking and
  substance involvement screening test (ASSIST).} Addiction 103.6
  (2008): 1039-1047.
\end{itemize}

\emph{February 8 - Practical Exercise 1 - Tentative Due Date}

\subsection{Week 06, 02/11 - 02/15: Psychological Assessment -
Behavioral
Data}\label{week-06-0211---0215-psychological-assessment---behavioral-data}

\begin{itemize}
\tightlist
\item
  Eyberg, Sheila M., and Elizabeth A. Robinson. \emph{Dyadic
  parent-child interaction coding system.} Seattle, WA: Parenting
  Clinic, University of Washington (1981).
\end{itemize}

\emph{February 15 - Practical Exercise 2 - Tentative Due Date}

\subsection{Week 07, 02/18 - 02/22: Psychological Assessment -
Psychometric
Tools}\label{week-07-0218---0222-psychological-assessment---psychometric-tools}

\begin{itemize}
\item
  Bathurst, Kay, Allen W. Gottfried, and Adele E. Gottfried.
  \emph{Normative data for the MMPI-2 in child custody litigation.}
  Psychological Assessment 9.3 (1997): 205.
\item
  Azar, Sandra T., Michael T. Stevenson, and David R. Johnson.
  \emph{Intellectual disabilities and neglectful parenting: Preliminary
  findings on the role of cognition in parenting risk.} Journal of
  mental health research in intellectual disabilities 5.2 (2012):
  94-129.
\end{itemize}

\emph{February 22 - Practical Exercise 3 - Tentative Due Date}

\subsection{Week 08, 02/25 - 03/01: Investigative Assessment -
Maltreatment Risk
Assessment}\label{week-08-0225---0301-investigative-assessment---maltreatment-risk-assessment}

\begin{itemize}
\item
  Eubanks, V. \emph{Chapter 4: The Allegheny Algorithm}, in Automating
  inequality : how high-tech tools profile, police and punish the poor
  (2018). (Joe to provide a scan).
\item
  Cuccaro-Alamin, Stephanie, Regan Foust, Rhema Vaithianathan, and Emily
  Putnam-Hornstein. \emph{Risk assessment and decision making in child
  protective services: Predictive risk modeling in context.} Children
  and Youth Services Review (2017).
\item
  Johnson, Kristen, and Deirdre O'Connor. \emph{Post-implementation
  Examination of a Risk Assessment's Ability to Classify Families by the
  Likelihood of Subsequent Child Protective Services Involvement.}
  (2008).
\end{itemize}

\emph{March 1 - Practical Exercise 4 - Tentative Due Date}

\subsection{Week 09, 03/04 - 03/08: Developing an Evidence-Based
Investigative Assessment
I}\label{week-09-0304---0308-developing-an-evidence-based-investigative-assessment-i}

\emph{March 8 - Practical Exercise 5 - Tentative Due Date}

\subsection{Week 10, 03/11 - 03/15: Developing an Evidence-Based
Investigative Assessment
II}\label{week-10-0311---0315-developing-an-evidence-based-investigative-assessment-ii}

\emph{Readings forthcoming}

\subsection{Week 11, 03/18 - 03/22: Exam Week - No
Class}\label{week-11-0318---0322-exam-week---no-class}

\emph{March 22 - Final Practical Exercise - Tentative Due Date}




\end{document}

\makeatletter
\def\@maketitle{%
  \newpage
%  \null
%  \vskip 2em%
%  \begin{center}%
  \let \footnote \thanks
    {\fontsize{18}{20}\selectfont\raggedright  \setlength{\parindent}{0pt} \@title \par}%
}
%\fi
\makeatother
